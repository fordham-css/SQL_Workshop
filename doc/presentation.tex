\documentclass[aspectratio=169]{beamer}
\usepackage{hyperref}
\usepackage{mybeamerdark}
\usepackage{tikz}
\colorlet{lightgray}{gray!20}
\usetikzlibrary{shapes,positioning,calc}

\lstset{
    language=C++,
    tabsize=4, % tab space width
    showstringspaces=false, % don't mark spaces in strings
    numbers=left, % display line numbers on the left
    basicstyle=\ttfamily,
    numberstyle=\color{comment}\ttfamily, % Line numbers
    commentstyle=\color{comment}\ttfamily, % comment color
    otherkeywords={>,<,-,!,=,~}, % Color operators too
    morekeywords={>,<,-,!,=,~},
    keywordstyle=\color{keyword}\ttfamily, % keyword color
    stringstyle=\color{string}\ttfamily, % string color
    morecomment=[l][\color{keyword}]{\#},
    emph={int,char,long,float,double,unsigned,namespace,typename},
    emphstyle={\color{types}\ttfamily\textit},
    escapeinside={@!}{!@},
    captionpos=b
}


\title{Intro to SQL}
\subtitle{Focusing on Sqlite}
\author{Richard Morrill}
\institute{Fordham University CS Society}
\logo{\includegraphics[width=2cm]{css_logo_color.png}}
\date{Wednesday, February 26th, 2020}

\begin{document}
\begin{frame}
    \titlepage
\end{frame}

\begin{frame}
    \frametitle{Game Plan}

    \begin{itemize}
        \item Installing Sqlite
        \item Why you need SQL
        \item What is SQL
        \item What is Sqlite
        \item Basics of SQL syntax and database design
        \item How adding a SQL\footnote[frame]{If pronounced ``sequel'', ``a'' is the correct article.} database to an existing project can
              simplify your code
    \end{itemize}

\end{frame}
\begin{frame}[fragile]
    \frametitle{Let's get it Installed!}

    \begin{columns}
        \begin{column}{0.4\textwidth}
            \underline{Mac and Linux}\\
            \small{
                There's a 99\% chance it's already installed.  Open a terminal
                window and try \code{sqlite3}, and if that doesn't work,
                \code{sqlite}.  If for some reason your OS didn't come with it,
                install it at \url{https://www.sqlite.org}
            }
        \end{column}
        \begin{column}{0.4\textwidth}
            \underline{Windows}\\
            \small{
                You could have unknowingly installed Sqlite when installing
                something else.  Give it a try in Powershell or CMD, but it's likley
                you'll need to install Sqlite at: \url{https://www.sqlite.org}
            }
        \end{column}
    \end{columns}
    ~\\~\\
    If it works you will see:\\
    (The specific version does not matter.)
    \begin{verbatim}
SQLite version 3.30.0 2019-10-04 15:03:17
Enter ".help" for usage hints.
Connected to a transient in-memory database.
Use ".open FILENAME" to reopen on a persistent database.
sqlite>
    \end{verbatim}
\end{frame}
\begin{frame}
    \frametitle{What's the point of this, anyways?}

    Non-CS majors that don't know SQL say, ``Can't I just use a GUI tool like MS
    Access?''
    \vfill
    CS majors that don't know SQL say, ``There's plenty of tools in the language I'm already using, why do I need another one?''
\end{frame}

\begin{frame}
    \frametitle{The power of SQL}
    \begin{itemize}
        \item SQL is a \emph{declarative language}.\pause You don't need to
        worry about \emph{how} the database engine is going to do something,
        just tell it what you \emph{want it to do}.
        \pause
        \begin{itemize}
            \item SQL lets you think about your data on a higher level than if you
                  were dealing with it in objects and arrays.
            \item Learning to think \emph{declaratively} can help you write more
            readable code, even when working in languages other than SQL.
        \end{itemize}
        \pause
        \item Even if your language provides its own set of
        data-manipulation tools, they're probably just different names
        for things SQL already does.
        \pause
        \begin{itemize}
            \item If you learn to use the real thing, you'll be able to figure out
                  pretty much any pseudo-sql stuff you come across.
        \end{itemize}
        \pause
        \item SQL gets to the point.
    \end{itemize}
\end{frame}
\begin{frame}
    \frametitle{A Bit About SQL}
    \begin{itemize}
        \item \emph{Structured Query language}
        \pause
        \item Developed by IBM in the 1970s
        \pause
        \begin{itemize}
            \item Like other languages developed by committee, has its weird
            bits.
            \item Syntax isn't always consistent with itself. 
        \end{itemize}
        \pause
        \item The gold standard for managing \emph{relational databases}.
        \pause
        \begin{itemize}
            \item A bit of a victim of its popularity, SQL isn't always
            implemented identically across database software from different
            vendors.
            \item 90\% of syntax is the same, but some things you ``get away
            with'' on one database you won't in others.
        \end{itemize}
        \pause
        \item Technically a fully-fledged programming language, although rarely used this way\only<6->\footnote[frame]{With one important exception we'll see later.}.
    \end{itemize}
\end{frame}
\begin{frame}[fragile]
    \frametitle{Relational Databases}

    Very few pieces of data exist in a vacuum\\
    \pause
    \vfill
    \begin{tikzpicture}[relation/.style={rectangle split, rectangle split parts=#1, rectangle split part align=base, draw, anchor=center, align=center, text height=3mm, text centered}]
        \node (departmenttitle) {\textbf{Department}};

        \node [relation=3, rectangle split horizontal, rectangle split part fill={comment}, anchor=north west, below=0.6cm of departmenttitle.west, anchor=west] (department)
        {\underline{Name}%
        \nodepart{two}   Manager
        \nodepart{three} Building};
        
        \node [below=1.3cm of department.west, anchor=west] (employeetitle) {\textbf{Employee}};

        \node [relation=5, rectangle split horizontal, rectangle split part fill={comment}, anchor=north west, below=0.6cm of employeetitle.west, anchor=west] (employee)
        {\underline{Id Number}%
        \nodepart{two}   Name
        \nodepart{three} Email
        \nodepart{four}  Supervisor
        \nodepart{five} Department};

        \pause
        \draw[-latex] (employee.five south) -- ++(0,-0.5) -| ($(employee.five south) + (2,0)$) |- ($(department.one south) + (0.25,-0.50)$) -| ($(department.one south) + (0.25,0)$);
        \pause
        \draw[-latex] (employee.four south) -- ++(0,-0.5) |- ($(employee.one south) + (0.25,-0.5)$) -| ($(employee.one south) + (0.25,0)$);
    \end{tikzpicture}\\
    Insert about 100 more tables here \dots
\end{frame}
\begin{frame}
    \frametitle{Consistency}
    \begin{itemize}
        \item Consistency is key to making relational systems work smoothly.
        \begin{itemize}
            \item Imagine what would happen if half of the employees called
            their department ``facilities'', while the other half called it
            ``maintenance''.
        \end{itemize}
        \pause
        \item With many different sources of data and users, maintaining
        consistency can be very difficult.
        \pause
        \begin{itemize}
            \item You can educate users $\rightarrow$ Like that's gonna happen\dots
            \pause
            \item You can write code that ``gaurds'' your data store $\rightarrow$ Closer, but what if you make a mistake?
        \end{itemize}
        \pause
        \item SQL builds consistency protection directly into the data storage\only<5->\footnote[frame]{This is assuming the SQL database is set up properly.}.
        \pause
        \begin{itemize}
            \item Bad data is rejected\dots No Exceptions!
            \item Strict column types
            \item Foreign keys: Can't insert an employee if his department is non-existent.
            \item Triggers: Arbitrary code that can stop mistakes in their tracks.
        \end{itemize}
    \end{itemize}
\end{frame}
\begin{frame}[fragile]
    \frametitle{Basics of SQL: Making a Table}
    \begin{enumerate}
        \item Open a terminal.
        \item Navigate to a directory you can use for temporary files.
        \item Type in \code{\$ sqlite demo.db}
        \item On the \code{sqlite>} prompt that will come up, enter the commands below.
    \end{enumerate}
    IN SQL WE ALWAYS SHOUT!
    \begin{lstlisting}[language=SQL]
CREATE TABLE "employee" (
    "id" INTEGER PRIMARY KEY,
    "name" VARCHAR,
    "dob" VARCHAR,
    "joinDate" VARCHAR
);
    \end{lstlisting}
    To see if it worked:\\\code{.tables}\\\code{.schema "employee"}
\end{frame}
\begin{frame}[fragile]
    \frametitle{Basics of SQL: Inserting and Reading Data}
    To make things more readable:\\
    \code{.mode column}\\
    \code{.nullvalue NULL}\\
    \code{.headers on}\\~\\
    Add Some Data:
    \begin{lstlisting}[language=SQL]
INSERT INTO "employee"
    VALUES (12, "Sam Smith", "1980-05-06", "2008-06-22");
INSERT INTO "employee" (id, "dob")
    VALUES (44, "1968-04-18");   
    \end{lstlisting}
    Read It Back:
    \begin{lstlisting}[language=SQL]
SELECT * FROM "employee";
SELECT "id", "name" FROM "employee";
    \end{lstlisting}
\end{frame}
\begin{frame}[fragile]
    \frametitle{Basics of SQL: Inserting and Reading Data}
    \framesubtitle{WHERE Conditions and ORDER BY}
    By default, SELECT returns all rows. \pause While you can simply iterate
    through the results in your own code to get the ones you want, there is a
    better way:
    \begin{lstlisting}[language=SQL]
SELECT * FROM "employee" WHERE id = 12;
SELECT * FROM "employee" WHERE id = 44 LIMIT 1;
    \end{lstlisting}
    There are many different conditions available in SQL.
    \pause
    The database engine is also very efficient at sorting data:
    \begin{lstlisting}[language=SQL]
SELECT * FROM "Employee" ORDER BY "dob" DESC;
    \end{lstlisting}
    How could we get the oldest employee?
    \pause
    \begin{lstlisting}[language=SQL]
SELECT * FROM "employee" ORDER BY "dob" ASC LIMIT 1;
    \end{lstlisting}
    Note that there is a MAX() function in SQL, but it probably won't work the
    way you expect.
\end{frame}
\begin{frame}[fragile]
    \frametitle{Basics of SQL: Deleting Data}
    Deleting in SQL is pretty intuitive:
    \begin{lstlisting}[language=SQL]
DELETE FROM "employee";
    \end{lstlisting}
    What just got deleted?\\
    \pause
    If you want to delete specific records, you need to use WHERE:
    \begin{lstlisting}[language=SQL]
DELETE FROM "employee" WHERE id = 12;
    \end{lstlisting}
    \pause
    To delete a table, you use DROP not DELETE:
    \begin{lstlisting}[language=SQL]
DROP TABLE "employee";
    \end{lstlisting}
    The convention of using DELETE for \emph{data} and DROP for \emph{structures} is widely used in SQL.
\end{frame}
\begin{frame}
    \frametitle{Basics of SQL: Constraints}

    A constraint is anything you set up in that controls what data may be put where.
    \pause
    \begin{itemize}
        \item Primary Keys
        \begin{itemize}
            \item Must be unique
            \item \emph{Tables exist to store primary keys, the other fields describe the primary key.}
            \item Database engine optimizes queries involving primary key
        \end{itemize}
        \pause
        \item Foreign Keys
        \begin{itemize}
            \item The primary key\only<3->\footnote[frame]{Sometimes it can be
            another field, but this is not recommended} of another table, stored
            as a regular column in your table
            \item Database engine will enforce the validity of whatever value
            you attempt to insert (e.g. can't insert an employee with a
            nonexistent department)
            \item Advanced Topic: You can control what happens to dependant rows
            on delete
        \end{itemize}
        \pause
        \item Other Keys / Indices
        \begin{itemize}
            \item The terms ``key'' and ``index'' have overlapping meanings
            \item You can arbitrarily enforce uniqueness on any column
            \item You can tell the database engine to optimize queries involving
            any column
        \end{itemize}
    \end{itemize}
\end{frame}
\begin{frame}[fragile]
    \frametitle{Foreign Keys}
    First, enable foreign keys:
    \begin{lstlisting}[language=SQL]
PRAGMA foreign_keys = ON;
    \end{lstlisting}
    Let's make a table with a foreign key:
    \begin{lstlisting}[language=SQL]
CREATE TABLE customer (
    id INTEGER PRIMARY KEY,
    name VARCHAR,
    contact INTEGER,
    FOREIGN KEY (contact) REFERENCES employee(id)
);
    \end{lstlisting}
    \pause
    Now try to insert a customer whose contact has Id 99
\end{frame}
\begin{frame}
    \frametitle{Demo Database}
    We've barely scratched the surface of what SQL can do, but before you learn
    any more facts, you should get some practice employing what we just learned.\\~\\
    In this repo you'll find some demo code, and a testing database:\\
    \url{https://github.com/fordham-css/SQL_Workshop}\\
    Clone / download the repo, and we'll do some exercises with the testing database.
\end{frame}
\begin{frame}
    \frametitle{C / C++ Integration}

    Let's walk through the C code in the repo to see how you can integrate Sqlite into C or C++.

\end{frame}
\end{document}